%% Feel free to edit this file to fit your needs
\usepackage{listings}
\usepackage{etoolbox}


\definecolor{base0}{RGB}{131,148,150}
\definecolor{base01}{RGB}{88,110,117}
\definecolor{base2}{RGB}{238,232,213}
\definecolor{sgreen}{RGB}{133,153,0}
\definecolor{sblue}{RGB}{38,138,210}
\definecolor{scyan}{RGB}{42,161,151}
\definecolor{smagenta}{RGB}{211,54,130}


\newcommand\digitstyle{\color{smagenta}}
\newcommand\symbolstyle{\color{base01}}
\makeatletter %Allows the use of ampersand in a macro name giving access to lst@mod from package
\newcommand{\ProcessDigit}[1]
{%
  \ifnum\lst@mode=\lst@Pmode\relax%
   {\digitstyle #1}%
  \else
    #1%
  \fi
}
\makeatother %Returns ampersand to cat11 see: https://tex.stackexchange.com/questions/8351/what-do-makeatletter-and-makeatother-do

\lstdefinestyle{solarizedcsharp} {
  language=[Sharp]C,
  frame=lr,
  linewidth=160mm,
  breaklines=true,
  tabsize=2,
  numbers=left,
  numbersep=5pt,
  firstnumber=auto,
  numberstyle=\tiny\ttfamily\color{base0},
  rulecolor=\color{base2},
  basicstyle=\footnotesize\ttfamily,
  commentstyle=\color{base01},
  morecomment=[s][\color{base01}]{/*+}{*/},
  morecomment=[s][\color{base01}]{/*-}{*/},
  morekeywords={  abstract, event, new, struct,
                as, explicit, null, switch,
                base, extern, object, this,
                bool, false, operator, throw,
                break, finally, out, true,
                byte, fixed, override, try,
                case, float, params, typeof,
                catch, for, private, uint,
                char, foreach, protected, ulong,
                checked, goto, public, unchecked,
                class, if, readonly, unsafe,
                const, implicit, ref, ushort,
                continue, in, return, using,
                decimal, int, sbyte, virtual,
                default, interface, sealed, volatile,
                delegate, internal, short, void,
                do, is, sizeof, while,
                double, lock, stackalloc,
                else, long, static,
                enum, namespace, string, var},
  keywordstyle=\bfseries\color{sgreen},
  showstringspaces=false,
  stringstyle=\color{scyan},
  identifierstyle=\color{sblue},
  extendedchars=true,
  captionpos=b,
  linewidth=0.99\textwidth,
  xleftmargin=.1\textwidth,
  literate=
    {0}{{\ProcessDigit{0}} }1
    {1}{{\ProcessDigit{1}} }1
    {2}{{\ProcessDigit{2}} }1
    {3}{{\ProcessDigit{3}} }1
    {4}{{\ProcessDigit{4}} }1
    {5}{{\ProcessDigit{5}} }1
    {6}{{\ProcessDigit{6}} }1
    {7}{{\ProcessDigit{7}} }1
    {8}{{\ProcessDigit{8}} }1
    {9}{{\ProcessDigit{9}} }1
    {\}}{{\symbolstyle{\}} } }1
    {\{}{{\symbolstyle{\{}} }1
    {(}{{\symbolstyle{(}} }1
    {)}{{\symbolstyle{)}} }1
    {=}{{\symbolstyle{$=$}} }1
    {;}{{\symbolstyle{$;$}} }1
    {>}{{\symbolstyle{$>$}} }1
    {<}{{\symbolstyle{$<$}} }1
    {\%}{{\symbolstyle{$\%$}} }1,
}

\lstset{escapechar=@,style=solarizedcsharp}
